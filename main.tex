\documentclass[aspectratio=169]{beamer}
\usepackage[utf8]{inputenc}
\usepackage{multicol}

\usepackage{polski}
\usepackage{graphicx}
\usepackage{listings}

\usepackage[T1]{fontenc}

\definecolor{dark-gray}{gray}{0.05}

\title{Konteneryzacja z użyciem \textbf{Docker}'a \\Izolacja środowisk}
\date{\today}
\author[Jaworski]{Konrad Jaworski\\216782@edu.p.lodz.pl}

\usetheme{material}

\useDarkTheme
\usePrimaryBlue
\useAccentGreen

\begin{document}

\begin{frame}
    \titlepage
\end{frame}

\begin{frame}{Spis treści}
    \begin{card}
        \tableofcontents
    \end{card}
\end{frame}

\section{Wporwadzenie}
\begin{frame}{Czym jest docker?}\subsection{Czym jest docker?}

    \centering
    \includegraphics[scale=0.2]{img/logo.png}
    \bigskip

    \begin{card}
        Docker jest otwartoźródłowym oprogramowaniem, które służy do wirtualizacji na poziomie systemu operacyjnego.\\\\
        Dostarcza pakiety z oprogramowaniem uruchamiane w wyizolowanym środowisku - \textbf{kontenerze}.
    \end{card}
\end{frame}




\subsection{Czym są kontenery?}
\begin{frame}{Czym są kontenery?}
    \begin{card}
        \textbf{Kontener} (\textit{ang.} container) to jednostka oprogramowania opakowywująca kod i wszystkie jego zależności w taki sposób, aby aplikacje działały szybko, niezawodnie i \textbf{jednakowo} niezależnie od środowisk w jakim znajduje się kontener.
    \end{card}

    \begin{card}
        \textbf{Obraz Kontenera} (\textit{ang.} container image) jest lekkim, samodzielnym i wykonywalnym pakietem oprogramowania, które zawiera wszystko co jest potrzebne do uruchomienia aplikacji: kod,
        środowisko uruchomieniowe (runtime), narzędzia systemowe, biblioteki i ustawienia.
    \end{card}
\end{frame}

\begin{frame}
    \begin{multicols}{2}
        \cardImg{img/what-is-container.png}{0.7\textwidth}
        \vspace*{47mm}
        \begin{card}
            Obraz kontenera staje się kontenerem w momencie uruchomienia obrazu w \textbf{Docker Engine}.
        \end{card}
    \end{multicols}
\end{frame}

\subsection{Docker Engine i jego komponenty}
\begin{frame}{Docker Engine}

    \begin{card}
        Docker Engine to aplikacja client-server, której główne komponenty to:
        \begin{itemize}
            \item Server, który działa jako usługa (\textit{daemon}) - \colorbox{dark-gray}{\lstinline{dockerd}}
            \item REST API, który służy aplikacjom do porozumiewania się z \textit{daemon'em}
            \item Interfejs użytkownika (\textit{ang. CLI - command line interface}) - klient \colorbox{dark-gray}{\lstinline{docker}}
        \end{itemize}
    \end{card}
\end{frame}

\begin{frameImg}[50mm]{img/architecture.png}
\end{frameImg}


\begin{frame}{Server - docker \textit{daemon}}
    \begin{card}
        Usługa servera docker'a (\colorbox{dark-gray}{\lstinline{dockerd}}) słucha zapytań API i na ich podstawie zarządza \textbf{obiektami docker'a} - obrazami (images), kontenerami(containers), sieciami (networks) oraz woluminami (volumes).\\\\
        Usługa ta potrafi również komunikować się z innymi usługami docker'a (np. w celu zarządzania docker swarm).
    \end{card}
\end{frame}

\begin{frame}{Klient docker'a}
    \begin{card}
        Klient docker'a (polecenie \colorbox{dark-gray}{\lstinline{docker}}) jest głównym sposobem interakcji użytkownika z resztą oprogramowania docker'a.
    \end{card}
    \begin{card}
        Gdy wywołamy polecenie \colorbox{dark-gray}{\lstinline{docker run}}, klient wysyła instrukcje za pomocą API do usługi  \colorbox{dark-gray}{\lstinline{dockerd}}, która następnie je wykonuje. Klient docker'a może komunikować się z więcej niż jednym serverem.

    \end{card}
\end{frame}

\begin{frame}{Rejestry docker'a - Docker Hub}
    \begin{card}
        Rejestry docker'a służą do przechowywania obrazów. Najbardziej popularnym oraz publicznie dostępnym jest \textit{Docker Hub} (\url{https://hub.docker.com/}).
        Docker jest domyślnie skonfigurowany w taki sposób, żeby w pierwszym miejscu szukać obrazów właśnie w tym rejestrze. Możliwe jest jednak tworzenie własnych.
    \end{card}
    \begin{card}
        Gdy wywołamy polecenie \colorbox{dark-gray}{\lstinline{docker pull}}, wymagane obrazy są pobierane ze skonfigurowanego rejestru.\\
        Natomiast gdy użyjemy \colorbox{dark-gray}{\lstinline{docker push}}, nasz obraz zostaje wypchnięty do skonfigurowanego rejestru.
    \end{card}
\end{frame}

\subsection{Obiekty docker'a}
\begin{frame}{Obrazy docker'a (\textit{Docker images})}
    \begin{card}
        Obraz (\textit{image}) jest wzorcem (tylko do odczytu) zawieracjącym instrukcje potrzebne do stworzenia kontenera docker'a.
        Często obrazy są bazowane na innych, ale zawierają pewne zmiany - na przykład możemy zbudować obraz bazujący na obrazie \colorbox{dark-gray}{\lstinline{ubuntu}},
        ale możemy na nim zainstalować dodatkowo Web Server Apache oraz skonfigurować go zgodnie z naszymi potrzebami.
    \end{card}
    \begin{cardTiny}
        Możemy korzystać z obrazów stworzonych przez innych, ale możemy również tworzyć własne obrazy, a następnie publikować je w rejestrach.
        Aby stworzyć własny obraz, należy stworzyć \textbf{Dockerfile} - instrukcje z prostą składnią, definiujące kroki potrzebne do stworzenia
        obrazu. Gdy zmodyfikujemy \textit{Dockerfile}, a następnie ponownie zbudujemy obraz, zmienią się jedynie te warstwy, które zmodyfikowaliśmy.
    \end{cardTiny}

\end{frame}

\begin{frame}{Kontenery docker'a (\textit{Docker containers})}
    \begin{card}
        Kontener docker'a to wykonywalna instancja obrazu. \\
        Za pomocą API docker'a możemy wykonywać m.in. następujące operacje:
        \begin{itemize}
            \item tworzyć kontenery - \colorbox{dark-gray}{\lstinline{docker container create}}
            \item uruchamiać kontenery - \colorbox{dark-gray}{\lstinline{docker container start}}
            \item zatrzymywać kontenery - \colorbox{dark-gray}{\lstinline{docker container stop}}
            \item przenosić kontenery - \colorbox{dark-gray}{\lstinline{docker container move}}
            \item usuwać kontenery - \colorbox{dark-gray}{\lstinline{docker container delete}}
        \end{itemize}
    \end{card}
    \begin{cardTiny}
        Kontener możemy połączyć z jedną lub więcej sieci, podłączyć przestrzeń dyskową, a nawet tworzyć nowe obrazy na jego podstawie.
    \end{cardTiny}
\end{frame}

\begin{frame}
    \begin{card}
        Już domyślnie kontenery są stosunkowo dobrze wyizolowane między sobą oraz między maszyną hosta.\\\\
        Można natomiast dalej kontrolować w jakim stopniu izolujemy sieć, przestrzeń oraz inne podsystemy naszego kontenera.
    \end{card}
\end{frame}

\begin{frame}{Przykład uruchomienia kontenera}
    \begin{cardTiny}
        Przykładowo możemy uruchomić \colorbox{dark-gray}{\lstinline{ubuntu}} w kontenerze oraz interaktywnie połączyć się z jego powłoką: \colorbox{dark-gray}{\lstinline{\$ docker run -i -t ubuntu /bin/bash}}
    \end{cardTiny}

    \begin{cardTiny}
        Gdy wykonamy powyższe polecenie, następuje:
        \begin{itemize}
            \item Jeżeli obraz ubuntu nie zostanie znaleziony lokalnie, zostanie on pobrany z rejestru (domyślnie z Docker Hub)
            \item Docker tworzy nowy kontener
            \item Docker alokuje system plików (read-write) do kontenera - pozwala to na tworzenie/modyfikowanie plików z lokalnej przestrzeni hosta
            \item Docker tworzy interfejs sieciowy łączący kontener z domyślną siecią - m.in. przypisuje mu adres IP. Domyślnie kontener może łączyć się z internetem poprzez sieć hosta
            \item Docker uruchamia kontener i wywołuje \colorbox{dark-gray}{\lstinline{/bin/bash}} w kontenerze
        \end{itemize}
    \end{cardTiny}
\end{frame}

\begin{frame}{Usługi (Docker Services)}
    \begin{card}
        Usługi (Services) pozwalają na skalowanie kontenerów z wykorzystaniem wielu współpracujących serwerów/\textit{daemon'ów} docker'a.
        Takie rozwiązanie nazywamy \textbf{\textit{Docker Swarm}} i składa się ono z \textit{managerów} i robotników (\textit{workers}). 
        Każdy członek roju (ang. \textit{swarm}) jest \textit{daemonem} dockera i komunikuje się z innymi członkami za pomocą API.
    \end{card}
\end{frame}


\begin{frameImg}[\textwidth]{img/engine-components-flow.png}
\end{frameImg}

\section{Podsumowanie}
\begin{frame}{Podsumowanie}
    \begin{card}
        Kontenery docker'a uruchamiane w Docker Engine są:
        \begin{itemize}
            \item Jednakowe: docker stworzył standard, dzięki któremu aplikacje będą działały tak samo niezależnie od środowiska
            \item Lekkie: kontenery dzielą kernel maszyny hosta, więc nie potrzebują swojego systemu operacyjnego, redukując używane zasoby
            \item Bezpieczne: docker dostarcza największą możliwą izolację pomiędzy kontenerami i hostem
        \end{itemize}
    \end{card}
\end{frame}



https://www.docker.com/why-docker
https://www.docker.com/resources/what-container
https://www.docker.com/products/container-runtime
% https://docs.docker.com/get-started/overview/#docker-architecture
https://docs.docker.com/engine/


\end{document}
