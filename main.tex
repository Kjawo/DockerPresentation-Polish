\documentclass[aspectratio=169]{beamer}
\usepackage[utf8]{inputenc}
\usepackage{multicol}

\usepackage{polski}
\usepackage{graphicx}

\usepackage[T1]{fontenc}

\title{Konteneryzacja z użyciem \textbf{Docker}'a \\Izolacja środowisk}
\date{\today}
\author[Jaworski]{Konrad Jaworski\\216782@edu.p.lodz.pl}

\usetheme{material}

\useDarkTheme
\usePrimaryBlue
\useAccentGreen

\begin{document}

\begin{frame}
    \titlepage
\end{frame}

\begin{frame}{Spis treści}
    \begin{card}
        \tableofcontents
    \end{card}
\end{frame}

\section{Wporwadzenie}
\begin{frame}{Czym jest docker?}\subsection{Czym jest docker?}

    \centering
    \includegraphics[scale=0.2]{img/logo.png}
    \bigskip

    \begin{card}
        Docker jest otwartoźródłowym oprogramowaniem, które służy do wirtualizacji na poziomie systemu operacyjnego.\\\\
        Dostarcza pakiety z oprogramowaniem - \textbf{kontenery}.
    \end{card}
\end{frame}




\subsection{Czym są kontenery?}
\begin{frame}{Czym są kontenery?}
    \begin{card}
        Setup is really easy:
        {\color{primary}\textbackslash usetheme\{material\}}
    \end{card}
    \begin{card}
        Further you might want to customize the background with: \\[2mm]
        {\color{primary}\textbackslash useLightTheme} or {\color{primary}\textbackslash useDarkTheme} \\[2mm]
        and primary and accent colors.
    \end{card}
\end{frame}

\subsection{Kontenery, a maszyny wirtualne}
\begin{frame}{Kontenery, a maszyny wirtualne}
    \begin{card}
        Containers and virtual machines have similar resource isolation and allocation benefits, 
        but function differently because containers virtualize the operating system instead of hardware. 
        Containers are more portable and efficient.
    \end{card}
\end{frame}

\end{document}
